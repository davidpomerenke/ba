\section{The intuition of neutrality}

The intuition of neutrality is assumed to be a widespread ethical intuition among humans \cite[p.~176f]{broome_2012}. The content of the intuition is called the principle of equal existence \cite[p.~146]{broome_2004}, but usually (and also in this thesis) the term "intuition" is also used to refer to the content of the intuition. 

The content of the intuition is defined as follows: Let us assume two hypothetical scenarios A and B. The same people exist in both scenarios, except that in scenario B there are some additional people which do not exist in scenario A. The intuition says: Which one of the scenarios is better depends entirely on the well-being of the people who exist in both scenarios, and not at all on the additional people who only exist in B – as long as all the additional people in B have a well-being within a certain neutral range. More specifically, as long as the additional people in B are within the neutral range, scenario A is better in terms of welfare if the people who exist in both populations have a higher welfare in scenario A, and scenario B is better in terms of welfare if the people who exist in both populations have a higher welfare in scenario B.  

We can formalize the scenarios as different welfare distributions represented by the welfare functions $u_A$ and $u_B$. Let $P_0$ be a population of people who exist in both scenarios but need not have the same levels of well-being in both scenarios. Let $P_+$ be the population of people who exist only in scenario B. Let $[u_1, u_2]$ be the neutral range of well-being for added people. 

\begin{Definition}{The intuition of neutrality}{ion} 
$
  \exists u_1, u_2: 
$

\hspace{.5cm} $
  (\forall x \in P_+ : u_B(x) \in [u_1, u_2] )
  \rightarrow 
$

\hspace{1cm} $
  ( u_B(P_0) > u_A(P_0) 
  \rightarrow 
  u_B(P_0 \cup P_+) > u_A(P_0) ) 
  \ \wedge
$  

\hspace{1cm} $
  ( u_B(P_0) < u_A(P_0) 
  \rightarrow 
  u_B(P_0 \cup P_+) < u_A(P_0) ) 
$
\end{Definition}

The formalization is to be interpreted in the following way: It does not matter in terms of welfare whether there exists an additional person in the population who lives at a moderate level of well-being. There are several moderate levels of well-beings, which form a range between a low moderate level of well-being $u_1$ and a high moderate level of well-being $u_2$. If however the additional person is at a very low level of well-being – below $u_1$ – then the person might matter for the calculation of general well-being. (Arguably, the welfare would decrease because of the added person; though this is not specified by the intuition.) Similarly, if the additional person is at a very high level of well-being – above $u_2$ – then the person might matter for the calculation of general well-being. (Arguably, the welfare would increase because of the added person.) 

There is a variation of the intuition of neutrality where the neutral range has no upper limit, i. e., $u_2=\infty$ \cite[p.~113]{broome_2012}. This may be a better representation of common belief, and I will come back to it in \todo{some section}. Whether the upper limit of the range is finite or infinite is of minor concern for this thesis; it is more important to note that there is \emph{some} neutral range. 

If the range is sufficiently large, this might simplify welfare calculations, as the following examples demonstrate:

\begin{itemize}
\item An exemplary application of the intuition is the evaluation of road safety \cite[p.~144f]{broome_2004}. In this context, the deaths of people dying in accidents must be weighed against the costs of preventing them. Whilst this is an ethically difficult problem on its own, one important long-term effect is usually left aside: The well-being of the expected potential offspring of the potentially dying person is completely neglected. One possible justification is the intuition of neutrality: According to the intuition, if we can expect the offspring to live within the neutral range of well-being, it is neither positive nor negative whether they exist or not.
\item A second example is the evaluation of different scenarios of global warming \cite[p.~170]{broome_2012}. Global warming is likely to kill many people and thereby to prevent their offspring from existing. On the other hand, global warming may increase poverty, which is associated with higher birth rates. Thanks to the intuition of neutrality we can simply leave both of these effects aside in many of our evaluations – which comes handy as predictions in these domains attend to an enormous amount of uncertainty. \citeNP[p.~120ff.]{broome_2012} sees massive problems if the intuition of neutrality cannot be assumed to apply.
\end{itemize}

It is important to understand that the intuition of neutrality does not imply neutrality about the consequences on the existing population which are caused by the additional population. These consequences may be negative or positive, leading to contrary political reactions such as China’s restrictive one-child-policy and Europe’s reproduction-promoting policy \cite[p.~169]{broome_2012}. The consequences on the existing population may well determine whether additional people are good or not. Only the well-being of the additional people themselves does not do so according to the intuition of neutrality. 

The question whether the intuition of neutrality is in fact a widespread intuition among humans appears not to have been investigated. It is not necessary for the argument against neutrality to assume such an empirical fact. Neither is it necessary for the refutation of this argument to assume so. If however this refutation were successful and the integrity of the intuition thus restored, then it would be desirable to investigate the empirical prevalence of the intuition. 