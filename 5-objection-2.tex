\section{The neutral range}
\label{sec:obj2}

Whilst section \ref{sec:obj1} has dealt with the implications of assuming the difference principle as a welfare function, this section deals with a second popular welfare function, that is average utilitarianism. 

I will start by explaining how the argument against neutrality requires the neutral range to be a proper range rather than a single level. Afterwards, I will try to make plausible why (assuming average utilitarianism) we should rather assume a single neutral level in theory and elucidate how, taking uncertainty into account, this single neutral level may approach a proper neutral range in practice.  

So far, the formalization of the intuition of neutrality involves a neutral range $[u_1, u_2]$ without specifying $u_1$ or $u_2$. As per Definition \ref{def:ion}, the neutral range could in fact just be a single number with $u_1 = u_2$; but the interpretation of the intuition of neutrality tells us that this range is in fact supposed to be a proper range and rather large. 

The argument against neutrality, however, could be misunderstood as an argument against any kind of neutral range. The superficial reader -- understanding that the argument denies the possibility of a neutral range -- may suppose that it denies the possibility of \emph{any} neutral range. I want to ward off this potential misunderstanding: As demonstrated below, the argument against neutrality only denies the possibility of a \emph{proper} real range, that is, it denies that the intuition of neutrality holds for $u_1 \neq u_2$. 

A careful analysis of the argument against neutrality yields that it interprets the intuition of neutrality in a way that does not permit that the neutral range is just a single level of well-being: In order to neutralize and counter the positive difference in well-being for person p between scenarios B and C, the difference in well-being between scenarios B and C for person q must be negative. So the neutral range must allow for such a difference, because the well-being of both p and q is to be within the neutral range:

\begin{Corollary}{Proper neutral range}{}
\begin{enumerate}
\item[(C10)] \hspace{1cm} $(A2)  \wedge (A3) \Rightarrow u_B(p) > u_C(p)$
\item[(C11)] \hspace{1cm} $(C10) \wedge (A5) \Rightarrow u_B(q) < u_C(q)$
\item[(C12)] \hspace{1cm} $(C11) \wedge (A4) \Rightarrow u_1 < u_2$
\end{enumerate}
\end{Corollary}

Formally (C12), as an implication of the argument against neutrality, is a substantive specification of the intuition of neutrality. Contentwise (C12) is completely in line with the idea behind the intuition of neutrality (cf. \citeNP[p.~146]{broome_2004}): Added lives are neutral except if they are at a very low or very high level of well-being \cite[p.~172]{broome_2012}, so the neutral range is not only a proper range but also a rather big range. The crucial message from the Corollary is that the argument against neutrality has a hidden premise, which has not been made explicit so far: $u_1 < u_2$.

Even if the intuition of neutrality in this form empirically holds as a widespread intuition, it is theoretically problematic:
\begin{itemize}
\item One of its implications is for example that we cannot say that a scenario with many added people at the highest well-being within the neutral range is better than a scenario with many added people at the lowest level of well-being. This implication –- that well-being within the neutral range is incomparable –- is at least controversial. 
\end{itemize}
And there are other pressing theoretical questions: 
\begin{itemize}
\item What values should $u_1$ and $u_2$ assume? Imagine someone proposed as a specification that $u_1$ should be, say, at the level of well-being of the person at the top of the lowest 10\% of the population in terms of well-being. How should we respond? 
\begin{itemize}
\item How should we know whether that is correct?
\item What kind of arguments would we have to employ in order to plead for a higher or lower value?
\item What kind of ethical principle determines the range? 
\end{itemize}
\end{itemize}

These problems do not arise if we restrict the intuition of neutrality to a single level of neutral well-being: 
\begin{enumerate}
\item Such a restriction would directly invalidate the argument against neutrality and circumvent the problem of the incomparability of people within the neutral range which I have just touched upon. 
\item There exists an established ethical theory which justifies the existence of this level and explains what value it should take. 
\end{enumerate}
The theory in (2.) is average utilitarianism and one kind of justification for it is found in \citeNP{harsanyi_1955}. Average utilitarianism is a highly controversial theory, specifically but not only when it is understood as a complete moral theory rather than only a theory of goodness (cf. \citeNP[pp.~50-54]{broome_2012}; \citeNP[sec~2.1.1]{arrhenius_2017}; \citeNP[pp.~167-175, 572f]{rawls_2005}). But it is a popular and consistent ethical theory which is not only able to account for many other ethical intuitions but also to answer our quantitative and justificatory questions regarding the neutral level of well-being. The welfare function of average utilitarianism states that general welfare is the average of all individual well-being: 

\begin{Definition}{Average utilitarianism}{avg}
$u(P) = \sum_{p \in P} \frac{u(p)}{|P|}$
\end{Definition}

This implies that in order to be neutral to existing welfare, the welfare of an added population must equal the welfare of the existing population. Not every single added person needs to be at this neutral level, but rather the average of all added persons needs to be at this level. 

\begin{Definition}{The neutral range in average utilitarianism}{}
$[u_1, u_2] = u_1 = u_2 = u_0 = w(P)$
\end{Definition}

So average utilitarianism provides a response to the argument against neutrality by modifying the intuition of neutrality and assuming a neutral level instead of a neutral range. As a result, the intuition is consistent, calculable, and maybe even justified (regarding the justification, cf. \citeNP[pp.~161-175]{rawls_2005}; and also \citeNP{arrhenius_2017}). Average utilitarianism plays (in this case) a revisionist role, a theory of moral error (cf. \citeNP[p.~35]{mackie_1990}): It tells us to slightly adjust our intuition -- to sharpen it -- so that it is consistent in itself and in its relation to other moral judgments. This is an acceptable, maybe desirable intervention to the beliefs from our intuition. 

Furthermore, this theoretical sharpening would not even necessarily change our application of the intuition of neutrality. This is because in practice, uncertainties are attached to all quantities of well-being, specifically the neutral value. When I talk of 'uncertainty' here, then I refer to 'measurement uncertainty' as used in statistics and the quantitative sciences. (The uncertainty in question is quantifiable, so in it falls into the decision-theoretic category of risk and not into the decision-theoretic category of uncertainty.) Measurement uncertainty is a well-developed theory (see, e. g., \citeNP{runge_2007}). Unlike the approaches of introducing indeterminacy in the forms of incommensurateness or vagueness (cf. alternative (d) in section \ref{sec:arg}) -- which are pursued and discarded as a solution to the argument against neutrality in \citeNP[pp.~164-183]{broome_2004} -- uncertainty does not suffer from difficult problems such as greediness (a problem discussed in \citeNP[p.~170ff]{broome_2004}).  

The neutral value is affected by two kinds of uncertainty: 
\begin{enumerate}
\item The first kind of uncertainty arises from the definition of the neutral level. Sensitivity analysis (cf. \citeNP{runge_2007}) of Definition \ref{def:avg} tells us that the uncertainty of the neutral level is composed of the average uncertainty of the well-being of all existing people. 
\item The level of well-being of any actual person that is considered to be at the neutral level or not is also subject to some uncertainty. In both cases, the uncertainty arises from the difficulty to quantify the personal well-being of existing or hypothetical persons (see \citeNP[p.~317-319]{harsanyi_1955} for discussion). 
\end{enumerate}
These uncertainties are not on a theoretical level. On the theoretical level it has been questioned that such quantifications are metaphysically and psychologically possible at all (cf. \citeNP[pp.~317-319]{harsanyi_1955}). -- On the practical level, these quantifications are de facto happening, but there is a great level of uncertainty attached to them \cite[ch.~9]{broome_2004}. 

\begin{comment}
Person q from Proof \ref{prf:arg} is at different levels in scenario A and B. We can interpret the neutral range as a neutral level $u_0 = u_1 + (u_2 - u_1) / 2$ with an uncertainty of $\sigma (u_0) = \pm (u_2 - u_1) / 2$. As a consequence, the different values of well-being for person q in scenarios A and B are compatible with the assumption that there is a single neutral level with $u_0$ with an uncertainty of $\sigma(u0)$.
\end{comment}

We can then accept the theoretical notion of a neutral level while at the same time both maintaining the practical idea of the intuition of neutrality and avoiding the argument against neutrality. 

Is doing so just a sophisticated trick? No: The specific nature of the intuition of neutrality had not been analysed before. Rather, it may have been a bit rash to conclude from the rough idea of the intuition of neutrality that it has to be formalized as a proper range. 

This section has explained that there is at least no obvious possibility of justifying such a range, and that as a consequence we do not know how to quantify the range. Average utilitarianism presents a possible justification for a neutral level, and together with uncertainty it can justify something like a range. This formal interpretation may be even closer to the empirical intuition of neutrality than the interpretation as a real range is. If it is not, the problem of the incomparability within the neutral range and the problem of the quantifiability of the neutral range present compelling reasons why we should adjust our intuitions.  