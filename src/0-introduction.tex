\chapter{Introduction}

How should we as a society value changes in population size? The question may be crucial when evaluating global warming scenarios. I defend the intuition of neutrality, which answers a part of the question. It states that – other things being equal – it is ethically irrelevant whether or not additional people are added to a population. The argument against neutrality criticizes the intuition to be inconsistent. I present three new objections to the argument: First, economic efficiency needs not be assumed as an ethical principle. Second, the intuition can be interpreted consistently in terms of uncertainty. Third, the intuition can be interpreted and justified in contractarianism. These objections are independent from each other. They are built on controversial philosophical views and do not necessarily disprove the argument against neutrality. Rather, they undermine the authority of the argument by pointing out the weakness of several premises.  

I begin by briefly introducing the framework of welfare economics, which this essay argues within. I then present in more detail the intuition of neutrality and the formal argument brought forward against it. The main part is dedicated to the development of three objections to the argument. I conclude with some remarks about the status and plausibility of the different objections. 