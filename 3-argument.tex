\section{The argument against neutrality}

The argument against neutrality (\label{ref:RNDVrb2Fg6EjX}Broome 2012, p. 177f., where the figure below is also copied from) concludes that the intuition of neutrality is inconsistent. The argument is a version of the mere addition paradox (\label{ref:RNDn8ybKRhpdp}Broome 2004, p. 148) and a modification of the adoption problem (\label{ref:RND0coQvEEBxt}Broome 2004, p. 161). It is based on the following counter-example to the intuition of neutrality:  

\begin{center}
  \includegraphics[width=0.8\textwidth]{3-fig-1}
\end{center}

There are three scenarios A, B and C. They share the same population, except that one additional person exists in both B and C. In both B and C the additional person has a level of well-being within the neutral range. The argument is structured into two major steps:  

First, scenario A is being compared to scenario B and to scenario C. The additional person can be neglected in this step because the person is within the neutral range. There is one person who is a little bit better off in scenario B than in scenario A. As all other persons have exactly the same level of well-being, it is reasonable that there is a higher welfare in scenario B than in scenario A. Contrarily, there is one person who is a little bit worse off in scenario C than in scenario A. As all other persons have exactly the same level of well-being, it is reasonable that there is a higher welfare in scenario A than in scenario C. As a consequence of these two observations, scenario B has a higher welfare than scenario C. Technically, this conclusion requires transitivity of the betterness relation.  

Second, scenario B is compared directly to scenario C. Both scenarios comprise the same people, so there is no additional person in either scenario who could be neglected. The person who is not present in scenario A and has therefore been neglected above is much better off in scenario C than in scenario B. This big difference clearly outweighs the difference of the other person’s well-being in favour of scenario B. As there is moreover a higher equality in scenario C, scenario C obviously has a higher welfare than scenario B. This is in contradiction to the result of step one, so the counter-example refutes the intuition of neutrality, which has been its core assumption.  

Whilst the argument above is intuitively plausible, it has two other important premises (\label{ref:RNDcNWwD0bGcJ}Broome 2012, p. 177f): First, if in two scenarios all persons have the same level of well-being except for one person who is better off in the second scenario, then the welfare in the second scenario is higher than in the first. Technically, the second scenario Pareto dominates the first (\label{ref:RNDe8Jgwq2mlh}Osborne 1997). Second, if in two scenarios with the same population the sum of individual well-being is higher in the second scenario, and at the same time the inequality of the distribution of well-being is lower in the second scenario, then the second scenario is better in terms of welfare than the first. I call this the fair aggregation principle. There are various ways to measure inequality, and the details need not concern us here. An excellent survey of one-dimensional inequality measures – as applied in welfare economics – is given in \label{ref:RNDgO1lcEIWbu}Sen \& Foster 1997. The most prominent inequality measure is probably the Gini coefficient (see \label{ref:RNDJkmhvOr2SL}Ceriani \& Verme 2012). Both of these premises appear to be very plausible, and they are dubbed “hard-to-doubt assumptions” in \label{ref:RNDUBSa4lnZMJ}Broome 2012, p. 176.  

The following formalization concisely sums up the argument presented above. It is structured in technical premises (P...), the assumptions that make up the setting of the counter-example (A...), and the conclusions drawn (C...). We can infer from the contradiction in (C8) that at least one of the premises and assumptions must be false. The assumptions merely describe the setting of the scenarios as depicted in the figure above. They are simply the assumptions making up the counter-example and there is no reason to doubt them within this proof. Moreover, premises (P2) and (P3) appear to be very plausible. As a consequence, the intuition of neutrality must be the false premise.  

Proof 1: The argument against neutrality 
\begin{comment}
\begin{flushleft} \tablefirsthead{} \tablehead{} \tabletail{} \tablelasttail{} \begin{supertabular}{m{1.201cm}m{13.639cm}} (P1) & Intuition of neutrality (see Def. 2)\\ (P2) & Transitivity of betterness: 

uX(PX) {\textgreater} uY(PY) ${\wedge}$ uY(PY) {\textgreater} uZ(PZ) \ding{213} uX(PX) {\textgreater} uZ(PZ)\\ (P3) & Pareto domination: 

\begin{enumerate} \item ( ${\exists}$ p ${\in}$ P :\ \  \ uX(a) {\textgreater} uY(p) \ ${\wedge}$  \end{enumerate} \ \ ${\forall}$ q ${\in}$ P ${\setminus}$ \{p\} : uX(q) = uY(q) )  

\ding{213} \ \ uX(P) {\textgreater} uY(P) 

\begin{enumerate} \item ( ${\exists}$ p ${\in}$ P :\ \  \ uX(a) {\textless} uY(p) \ ${\wedge}$  \end{enumerate} \ \ ${\forall}$ q ${\in}$ P ${\setminus}$ \{p\} : uX(q) = uY(q) )  

\ding{213} \ \ uX(P) {\textless} uY(P)\\ (P4) & Fair aggregation principle: 

${\sum}$p${\in}$PuX(p) {\textgreater} ${\sum}$p${\in}$PuY(p)  

${\wedge}$ gX(P) {\textless} gY(P)  

\ding{213} uX(P) {\textgreater} uY(P) 

with suitable inequality function g (see above).\\ (A1) & Let A, B, C be scenarios with  

corresponding distributions of well-being uA, uB, uC and  

populations PA = P0, PB = PC = P0 ${\cup}$ P+, and  

some person p ${\in}$ P0  

such that uA(P0 ${\setminus}$ \{p\}) = uB(P0 ${\setminus}$ \{p\}) = uC(P0 ${\setminus}$ \{p\}). \\ (A2) & Let uB(p) {\textgreater} uA(p).\\ (A3) & Let uC(p) {\textless} uA(p).\\ (A4) & Let P+ = \{q\} with uB(q), uC(q) ${\in}$ [u1, u2]\\ (A5) & Let uB(p) + uB(q) {\textless} uC(p) + uC(q).\\ (A6) & Let gB(P0 ${\cup}$ P+) {\textgreater} gC(P0 ${\cup}$ P+). 

––––––––––––––––––––––––––––––––––––––––––––––\\ (C1) & (P3) ${\wedge}$ (A1) ${\wedge}$ (A2) \ $\Rightarrow $ \ uB(P0) {\textgreater} uA(P0)\\ (C2) & (P3) ${\wedge}$ (A1) ${\wedge}$ (A3) \ $\Rightarrow $ \ uC(P0) {\textless} uA(P0)\\ (C3) & (C1) ${\wedge}$ (P1) ${\wedge}$ (A4) \ $\Rightarrow $ \ uB(P0 ${\cup}$ P+) {\textgreater} uA(P0)\\ (C4) & (C2) ${\wedge}$ (P1) ${\wedge}$ (A4) \ $\Rightarrow $ \ uC(P0 ${\cup}$ P+) {\textless} uA(P0)\\ (C4) & (C3) ${\wedge}$ (C4) ${\wedge}$ (P2) \ $\Rightarrow $ \ uB(P0 ${\cup}$ P+) {\textgreater} uC(P0 ${\cup}$ P+)\\ (C6) & (A1) ${\wedge}$ (A5) \ $\Rightarrow $ \ ${\sum}$x${\in}$P0uB(x) {\textless} ${\sum}$x${\in}$P0uC(x)\\ (C7) & (C6) ${\wedge}$ (P4) ${\wedge}$ (A6) \ $\Rightarrow $ \ uB(P0 ${\cup}$ P+) {\textless} uC(P0 ${\cup}$ P+)\\ (C8) & (C4) \ $\Leftrightarrow $ \ ¬ (C7) \\ (C9) & (C8) ${\wedge}$ (A1-A6) ${\wedge}$ (P2) ${\wedge}$ (P3) ${\wedge}$ (P4) \ $\Rightarrow $ \ ¬ (P1)\\ \end{supertabular} \end{flushleft}
\end{comment}

There are two implications if this argument holds and the intuition of neutrality is inconsistent (cf. \label{ref:RNDvHVBymXoLa}Broome 2005, p. 411): First, we as a society would have to develop a different, consistent principle to replace the intuition. We do not even currently know whether population changes should be evaluated as positive or as negative, just that they cannot simply be evaluated as neutral. The finding of a new principle with wide acceptance would certainly present a major societal task and require many years of discourse. Second, we would need to gain better knowledge of which actions lead to which consequences with respect to population changes. Only then would we probably be able to apply a principle which is not based on neutrality. This requires new scientific analysis and simulation because such predictions have often been omitted in the past (\label{ref:RND54So3qk8SK}Broome 2005, p. 402; \label{ref:RNDJG5XuCbvIG}Broome 2012, p. 115f).  

\label{ref:RNDoGhae544eI}Broome 2004 develops five possible modifications to respond to the argument against neutrality (see p. 149):  

\begin{enumerate} \item intransitivity of the betterness relation \item conditional goodness \item relative goodness \item indeterminacy or vagueness of the betterness relation \item a single neutral level \end{enumerate} The transitivity of the betterness relation is plausibly defended in \label{ref:RNDLa97xVELwW}Broome 2004, p. 151f. I will pursue a somewhat related approach to (b) and (c), focused more on justification, in the third objection. The second objection will be very similar to what is developed in (d) but it will also be compatible with (e). Whilst (P1) is discussed in (a), (P2) and (P3) have not been discussed so far. This is what I will do in the first objection.  