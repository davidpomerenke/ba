\section{Relativism}
\label{sec:obj3}

While sections \ref{sec:obj1} and \ref{sec:obj2} examine how two different welfare functions affect the result of the comparison of scenarios with a changing population, this section investigates which persons are to be regarded in such comparisons. We will see that the argument against neutrality makes some controversial implicit assumptions about which persons are to be regarded; a modification of these assumptions may therefore avoid the argument against neutrality, regardless of the chosen welfare function. 

The difference principle and average utilitarianism are probably the two most prominent welfare functions. In both of them the argument against neutrality does not hold for different reasons. Yet, although these frameworks are so well received, they both raise the question of how they should be justified. The specific justificatory problem which matters in our context is that the frameworks assume a universal moral domain. This means that they assume that in the first place every person should receive moral consideration. If the universal domain is rejected, the results of sections \ref{sec:obj1} and \ref{sec:obj2} will no longer matter. We will now see that the argument against neutrality fails if the universal domain is rejected, and afterwards we will discuss whether such a relativistic position is acceptable. In the context of this section, 'relativism' refers to the procedure of undertaking ethical evaluations with respect to a certain (temporally limited) population rather than with respect to all persons ever existing. In a relativistic interpretation, 'goodness' is to be understood as 'goodness for someone or some group', and in our context usually as 'goodness for the current (world) population'. This form of relativism is more precisely referred to as \emph{metaethical relativism} (cf. \citeNP[ch.~2e]{westacott_2019}; \citeNP[ch.~2]{gowans_2019}).

So far we have interpreted the intuition of neutrality as a principle which is applied only in a particular instance of comparisons between welfare in different scenarios: Whenever there are additional persons in one scenario \emph{who do not exist in the other scenario,} then we can apply the intuition of neutrality. This is reflected in the reasoning in Proof \ref{prf:arg}: In (C3) and (C4) we have used the intuition of neutrality because there is a different number of persons in scenario A than there is in scenarios B and C. But in (C7) we have been comparing scenarios B and C, and these scenarios have the same number of persons, so we had to take into account the well-being of all people: "B and C contain the very same five people, so in comparing their values all five count as existing people." (\citeNP[p.~177]{broome_2012})

A simple escape from the argument against neutrality would be to deny that in such cases all people count as existing people. If we regard person q (the person who exists in scenarios B and C only) as non-existent, then we cannot derive that C is better than B by direct comparison (C7), and the argument fails: 
\begin{Proof}{Failing modification of the argument against neutrality}{mod}
Assume (C1) -- (C6) from proof \ref{prf:arg}. Assuming that q is disregarded in the comparison between B and C, we do not arrive at (C7), (C8), and consequently (C9); but rather at the different conclusions (C7') and (C8') from which the result of (C9) cannot be deduced. This means that the proof fails under the modified assumption. \\
That q is disregarded from the comparison between B and C formally follows from (P5): \emph{'A is the base scenario.'} This additional premise will be introduced and discussed below. 
\begin{enumerate}
\item[(C7')] \hspace{1cm} $(A1) \wedge (A2) \wedge (A3) \wedge (P1) \wedge (P5) \\ 
\hspace*{1cm} \Rightarrow  u_B(P_0 \cup P+) > u_C(P_0 \cup P_+)$
\item[(C8')] \hspace{1cm} $(C4) \Leftrightarrow (C7')$ 
\end{enumerate}
\end{Proof}

If we want to disregard person q in the comparison between scenarios B and C, we need a revised version of the intuition of neutrality (Definition \ref{def:ion}; Premise \ref{pre:p1}). The differences to the original formulation in Definition \ref{def:ion} are highlighted in yellow. 

\begin{Definition}{The intuition of neutrality (revised)}{mod} 
$
  \exists u_1, u_2: 
$

\hspace{.5cm} $
  (\forall x \in P_+ : u_B(x) \in [u_1, u_2] )
  \rightarrow 
$

\hspace{1cm} $
  ( u_B(P_0) > u_A(P_0) 
  \rightarrow 
  u_B(P_0 \cup P_+) > u_A(P_0 \highlight{ \cup P_+}) ) 
  \ \wedge
$

\hspace{1cm} $
  ( u_B(P_0) < u_A(P_0) 
  \rightarrow 
  u_B(P_0 \cup P_+) < u_A(P_0 \highlight{ \cup P_+})) 
$
\end{Definition}

In the definition above, the intuition has been modified in a way such that it also applies to comparisons where the persons are the same in both scenarios (such as in the comparison between scenarios A and B). As shown in \ref{prf:mod}, this avoids the argument against neutrality. 

But while this solution may be compelling so far, it brings with it a formal problem. In the original definition of the intuition of neutrality (Definition \ref{def:ion}), we have not really needed to specify $P_0$ and $P+$: $P_0$ has been the population which exists in both scenarios and $P_+$ has been the population which exists only in scenario B. This is no longer implicit in Definition \ref{def:mod}: $P_0$ and $P_+$ both exist in both scenarios; they are not distinguished by the definition. $P_+$ are the people who are neutral with respect to general welfare if their well-being lies within the neutral range; and $P_+$ are the same people in B and in C. But $P_+$ could be any persons: $P_+$ could be all persons, no persons, or an arbitrary selection of persons. So as they are not already formally specified we need to specify $P_+$:

\begin{Definition}{Additional specification of the intuition of neutrality}{add}
$P_0$ are the existing people and $P_+$ are the non-existing people.
\end{Definition}

This definition is completely in the spirit of the original intuition of neutrality: The intuition is originally about existing and non-existing people. 

Unlike all the formal definitions above, this is a material definition, which is not a problem. The problem is that it is also a time-relative definition. Which people are existing and which are not depends on the time of evaluation. 

\begin{Example}{Time-relativity}{}
When we consider whether it is good or not that a baby is born, we arrive at different evaluations before and after the pregnancy of the baby's parent. Before the pregnancy, the baby's well-being has to be ignored because of the intuition of neutrality, but after the pregnancy, the baby's well-being has to be considered. Imagine that we want to know whether it is positive or negative for the general welfare whether the baby suffers from a chronic disease. Then before the pregnancy we will derive that the chronic disease is neutral with respect to general welfare and after the pregnancy we will derive that it would be better for general welfare if the baby did not suffer from the disease. 
\end{Example}

The evaluation of welfare depends on what scenario we use as a base scenario based on which we judge which persons are existent and which are not. (This has been suggested by Stefan Fischer in a discussion.) Such a base scenario may be either of the scenarios which we compare, or a third scenario. In the case of the argument of neutrality, we need to choose scenario A as our base scenario so that we can arrive at the alternative conclusion (C8’). 

\begin{Premise}{Base scenario (P5)}{base}
A is the base scenario, with the corresponding distribution $u_A$ of well-being.
\end{Premise}

The additional specification and the additional premise are a form of metaethical relativism: We have to choose a base scenario, for which the ethical evaluation is conducted. This scenario will usually be our own, current scenario, because we are interested in persons which are added to our own, current population. $P_0$ will thus be the current world population, and $P_+$ will be some potential additional future population. If the question wil be discussed again in ten years, then the society existing in ten years will read $P_0$ as the population then existing, and $P_+$ as a potential additional future population that comes into being later than ten years from today. 

Moreover -- and now things are getting complicated -- when our society in ten years deliberates about the decisions taken today, they will have to employ their own standard from their own time: This means, they will count both the current population and all people coming into existence between now and ten years from now as $P_0$ when they want to arrive at their own normative evaluations. Only if they want to evaluate the current decisions from the perspective of the current population (which is usually a question of minor interest), the future population will, just like us today, use $P_0$ to refer to the current population. 

A similar approach to the described relativism is pursued in \citeNP[pp.~157-162]{broome_2004}. There it is discarded for two reasons:
\begin{enumerate}
\item Because of the incoherence which arises when switching the base scenario (pp. 68-76). 
\item Because of the difficulty to ethically justify person-relativity or com\-mu\-ni\-ty-relativity (p. 161f). 
\end{enumerate}
I will now address both issues.  

The problem of incoherence cannot be denied: If ethical evaluations of welfare depend on the choice of the base scenario and if every person chooses the person's own (current) situation as the base scenario, incoherence will arise. Principally, there are relativistic inconsistencies of several kinds. E. g., one person could contradict another person from the same population. As we are concerned with population ethics here, where persons will usually somehow consider the whole population for their evaluation, this is not necessarily a problem. A necessary problem is the time-dependence of the evaluation, which is pointed out in \citeNP[p.~75]{broome_2004}: 
\begin{quote}
"You choose rightly, but it later turns out you chose wrongly. Indeed, it may turn out that you ought later to undo what you rightly did. Moreover, you might be able to foresee even as you choose [your action] that just this would happen. This is a most implausible sort of incoherence in your activity."
\end{quote}

This criticism of time-relativity sounds like a problem at first, but I will now try to plausibilize that it might be acceptable. 

One possible defence of time-relativity is in denying that incoherence between actions is a problem at all. Incoherent actions are sometimes criticized in everyday situations; for example, when two persons or one person act out two actions which appear to follow opposite intentions, we might say that the actions are incoherent (or inconsistent). But the conceptual basis for such ascriptions is rather fuzzy and the existing theory of rationality does not provide a criterion for directly identifying incoherent actions. What the theory of rationality does provide is a criterion for identifying incoherent beliefs: Beliefs are incoherent if their propositional contents are contradictory. The only philosophically developed theory for criticizing inconsistent actions is (to my knowledge) to criticize inconsistencies in the belief set motivating the action (cf. \citeNP[ch.~4]{wilson_2016}). 

The underlying beliefs in a situation of alleged incoherence due to time-relativistic reasoning are complex: Before action A, we think that we should do A. And we think that as a causal effect of doing A, we will regret having done A. So we think that we should do A, and that we will regret it afterwards. After the action A, we think that we should not have done A. We also think that we have thought that we should do A. The point is that there is no apparent formal contradiction within this belief set. -- It may very well be rational to think A at the moment and to expect that oneself would think the opposite of A under different circumstances. As the enactment of A causes a change of circumstances, the complex belief set above may come about, and there is nothing wrong with it. 

Similarly, one may deny that it is irrational to \emph{undo} an action (cf. \citeNP[p.~75]{broome_2004}). A is done within one set of circumstances and then within another set of circumstances it is undone. The two actions of 'doing A' and 'undoing A' can be differentiated by the fact that they have taken place within different contexts (one context without the causal effects of A and one context with the causal effects of A). There is no obvious reason to ignore the contexts of the actions and to strip them down to the notions of 'doing A' and 'undoing A', creating a possibly false dichotomy. 

The above argumentation -- that the apparent incoherence due to temporal relativity is not irrational -- will certainly be highly controversial. But it is much more plausible that it applies to group beliefs and group actions. And it is group actions towards which the criticism of population-relativity in \citeNP[pp.~157-162]{broome_2004} is directed. 

\begin{Example}{Coherence of group actions}{}
As an illustration, imagine a direct democracy called Alphaland whose citizens consider -- say, because of their liberal ideal -- to invade and annex an autocratic country called Betaland with a population bigger than their own population. They consensually adopt a resolution: "Alphaland wants to annex Betaland. Alphaland expects Betaland to condemn the violence related to the annexation. As the current citizens of Betaland will be the majority in Alphaland after the annexation, we expect that Alphaland will officially regret the annexation afterwards. But we expect that most former Betaland citizens will nonetheless want to remain in Alphaland for pragmatic reasons (so the annexation would not be without effect)." It is plausible that rationality does not require present Alphaland to consider the interest of future Alphaland. After all, future Alphaland is made up of different citizens than present Alphaland. This example illustrates that there is no requirement for communities to be coherent with their beliefs and expected beliefs over time. 
\end{Example}

One might object to the above example: "If Alphaland was a single person, we would criticize the annexation as irrational." But this objection is beyond the point: Groups, though they may be identified -- analogous to single persons -- by names, do not possess beliefs in the same sense that persons do. This becomes apparent in situations like voting, where, for example, the voting paradox leads to counterintuitive conclusions (cf. \citeNP[ch.~1.1]{list_2013}). And group actions also belie one's expectations with respect to temporal coherence. Even if the rationality constraints of temporal coherence may apply for individual persons (which I have questioned above), they do not for groups. 

The previous paragraphs have been about the coherence of group beliefs. One might object: Are we not concerned with general ethical belief, which one would assume to be independent from the practical belief of some community? The answer is that in metaethical relativism all ethical evaluations depend on the fact which scenario is the base scenario, and the choice of the base scenario depends on the own, current population. 

After I have defended the form of metaethical relativism which comes along with Definitions \ref{def:mod}, \ref{def:add} and Premise \ref{pre:base}, I will end with some brief remarks regarding the justificatory status of metaethical relativism. Relativism is dropped in \citeNP[p.~160f]{broome_2004}, mainly because of (1) the alleged incoherence discussed above and (2) the lack of justification for the required form of metaethical relativism. Regarding (2), there are indeed good reasons why metaethical relativism is preferable to other preferable positions. I point towards the \emph{argument from queerness} and the \emph{argument from relativity} found in \citeNP[part~1]{mackie_1990} and summarized in \citeNP[supp.~2]{joyce_2016}. The contractarian tradition within ethics is closely associated with metaethical relativism (including time-relativism). One sophisticated contemporary ethical framework based on metaethical relativism is presented in \citeNP{stemmer_2013}. These references shall suffice to make clear that the justificatory status of metaethical relativism needs not be considered problematic, but to the contrary can be seen as very satisfactory. 

Even more than the arguments in sections \ref{sec:obj1} and \ref{sec:obj2}, the argumentation in this section touches upon several philosophical controversies, notably whether metaethical time-relativism is coherent and whether it is justified. The result is that if both questions can be answered with yes, the argument against neutrality is undermined. This is because metaethical time-relativism allows us to modify the intuition of neutrality in a way that is consistent with its original intention and which avoids the argument against neutrality by relying on a base scenario. 