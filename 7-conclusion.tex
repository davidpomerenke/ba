\chapter*{Conclusion}
\addcontentsline{toc}{chapter}{Conclusion}

\begin{adjustbox}{center}
\begin{tabular}{@{}llllll@{}}
\rot{Contractarianism} & \rot{Average Utilitarianism} & \rot{Difference Principle} & Note & Example & \rot{Compatible} \\ \midrule
\yo & \yo & \no & Utilitarianism with contractarian justification & \citeNP{harsanyi_1955} & \yo \\
\yo & \no & \yo & Difference principle with contractarian justification & \citeNP{rawls_2005} & \yo \\
\yo & \no & \no & Contractarianism with other or no welfare function & \citeNP{stemmer_2000} & \yo \\
\no & \yo & \no & Utilitarianism with other justification & \citeNP{mill_2016} & \yo \\
\no & \no & \yo & Difference principle with other justification & cf. \citeNP{pomerenke_2017} & \yo \\
\no & \no & \no & Other moral framework & ... & ? \\ \bottomrule
\end{tabular}
\end{adjustbox}
\begin{center}
\scriptsize Possible combinations of the three ethical frameworks of contractarianism (in a metaethically time-relative version), average utilitarianism, and the difference principle; along with a short characterization; one exemplary theoretical text; and the compatibility with the intuition of neutrality. \\
(\citeNP{mill_2016} is classified as average utilitarian in \citeNP[p;~38]{myrdal_2017}.)
\end{center}

This thesis has formally presented the framework of welfare economics, the intuition of neutrality, and the argument against (the intuition of) neutrality. It has proceded by critically analyzing the impact of three (meta)ethical frameworks on the soundness of the argument against neutrality, with the following results:
\begin{enumerate}
\item The difference principle in its Leximin formulation is in contradiction to the principle of fair aggregation, which is a premise for the argument against neutrality. The argument against neutrality therefore does not hold when assuming the difference principle. 
\item The argument against neutrality has the implicit premise that the neutral range within which persons can be added to a population is a proper range. Average utilitarianism suggests to slightly modify the intuition of neutrality so that it uses a neutral level rather than a proper neutral range. Because of the uncertainty involved, this is in practice somewhat similar to the original intuition of neutrality. The argument against neutrality therefore does not hold when assuming average utilitarianism and slightly adjusting the intuition of neutrality accordingly.
\item Assuming metaethical time-relativism (as is usually the case in contractarian theories), the intuition of neutrality can be modified by introducing a base scenario for comparisons. This modification is in line with the spirit of the original intuition of neutrality. The argument against neutrality does not hold when assuming this modification. 
\end{enumerate}
This means that in any of the three frameworks -- difference principle, average utilitarianism, implementations of metaethical relativism such as moral contractarianism -- the argument against neutrality does not hold. These frameworks do not present a complete partition of ethical beliefs, but they are major ethical frameworks. 

As a result, the intuition of neutrality \emph{can} be assumed in these frameworks. Whether the intuition of neutrality \emph{should} be adopted is left open in this thesis. Assuming that ethics is a system which should be built up from people's ethical convictions (a controversial claim on its own), an investigation into the empirical prevalence of the intuition should be conducted. If the intuition of neutrality is considered correct, it might simplify calculations with regard to pressing problems such as global warming. 