\chapter*{Conclusion}
\addcontentsline{toc}{chapter}{Conclusion}

\begin{adjustbox}{center}
\begin{tabular}{@{}llllll@{}}
\rot{Contractarianism} & \rot{Average Utilitarianism} & \rot{Difference Principle} & Note & Example & \rot{Compatible} \\ \midrule
\yo & \yo & \no & Utilitarianism with contractarian justification & \citeNP{harsanyi_1955} & \yo \\
\yo & \no & \yo & Difference principle with contractarian justification & \citeNP{rawls_2005} & \yo \\
\yo & \no & \no & Contractarianism with other or no welfare function & \citeNP{stemmer_2000} & \yo \\
\no & \yo & \no & Utilitarianism with other justification & \citeNP{broome_2004} & \yo \\
\no & \no & \yo & Difference principle with other justification & cf. \citeNP{pomerenke_2017} & \yo \\
\no & \no & \no & Other moral framework &  & N.N. \\ \bottomrule
\end{tabular}
\end{adjustbox}

\begin{itemize} \item explain why objection 3 is most important: 1 and 2 rely on rawls and avg util, which are basically based on an analytic explanation of – within our linguistic framework – justice as fairness and ethics as impartiality, resp., whilst 3 is based on the interest of people and thus normative \end{itemize} In summary, Broome commits two errors: In two common frameworks his “hard-to-doubt” strong Pareto assumption is false. In the third framework his implicit assumption that the intuition of neutrality requires proper ranges of neutral values is false. It is also plausible that the assumptions on their own do not reflect common ethical intuition. The intuition of neutrality might actually be right. 

\begin{itemize} \item point out conclusion: ion can maybe be held in utilitarianism (revision: no range but uncertainty), for sure in rawls \& similar frameworks, most plausibly in contractarianism (revision: no upper / lower boundaries) \item sum up frameworks, discuss whether they are a complete partitioning of ethical belief \item (Assuming that ethics is a system which should be built up from people’s ethical convictions,) we need empirical research on whether the intuition holds \item impact on climate change? we need not care about unborn people and can make more straight calculations \end{itemize}