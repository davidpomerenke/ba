\chapter{Exposition}
\section{Welfare Economics}

Welfare economics is the theory how individual well-being should be aggregated to general well-being (or welfare). General well-being drives decisions in the welfare state. The theory is relevant for the execution as well as the design of economic policies. As in democracies the citizens and their representatives take part in the design process, welfare economics is subject to societal discourse in these nations. Within this discourse, citizens and media often do not only claim their own interests. Instead they also refer to ethical principles which are to guide democratic policy decisions. This essay is set within this democratic discourse and aims to defend a supposedly widespread intuition whose consistency has been challenged from the academic side.  

The core of welfare economics is the welfare function (see \citeNP[p.~309]{harsanyi_1955}). It is an aggregation function: a function which takes in the individual levels of well-being of several individual persons, and delivers the level of welfare for the whole aggregated population comprising these individual persons. Well-being and welfare (which refers to aggregated well-being) are abstract terms. They are usually interpreted as a representation derived from a person's preferences about different lives (cf. \citeNP[ch.~4.2]{crisp_2017}). But they can also be interpreted simply as hedonic levels of lifetime pleasure \citeNP[ch.~4.1]{crisp_2017}, which will be sufficient for the purpose of this thesis. Well-being (or utility) of a person p is denoted by $u(p)$; individual persons are denoted by $p_i$ – the subscript is just there to differentiate between different persons. In similar fashion, welfare of a population $P = \{ p_1, p_2, …, p_n \}$ is denoted by $u(P)$. 

\begin{Definition}{Welfare Function}{}
$
  w: \hfill
  \mathbb{R}_+^{|P|} \rightarrow \mathbb{R}_+, \hfill
  \{ u(p1), u(p2), …, u(pn) \} \mapsto u( \{ p1, p2, …, p_n \} ) 
$
\end{Definition}

The content of the general welfare function is intentionally unspecified. The function is just a vehicle for discussion within welfare economics. Several specific welfare functions have been proposed and we will deal with two of them in later sections. For example, the classical utilitarian welfare function states that welfare is simply the sum $u(p_1) + u(p_2) + … + u(p_n)$ of all individual well-being.  

I introduce welfare functions because they are precise formalizations of competing ethical beliefs. In sections \ref{sec:obj1} and \ref{sec:obj2}, I will make use of them in order to demonstrate that when we assume certain ethical intuitions, the argument against neutrality does not hold. I will present two widespread competing ethical belief systems – average utilitarianism, and the difference principle – and try to refute the argument against neutrality from each of these views. The idea is that many people will adhere to one of these principles so that they can agree with at least one of the refutations. (Section \ref{sec:obj3} is of a different kind because, rather than to specific welfare functions, it relates to their justification.)

Welfare economics are blind in a certain respect, and so will be this discussion: They are consequentialist. This means that they only evaluate actions by their outcome and in this context specifically by their impact on general welfare or goodness. Other elements of ethical evaluation, such as the procedural requirements of justice, will have to be considered separately (cf. \citeNP[p.~401]{broome_2005}; \citeNP[p.~99f]{broome_2012}). These separate considerations will often require consequentialist considerations as part of their theoretical foundation, so this discussion may be indirectly relevant for them. 