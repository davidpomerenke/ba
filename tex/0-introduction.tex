\chapter*{Introduction}
\addcontentsline{toc}{chapter}{Introduction}

How should we as a society value changes in population size? The question may be crucial when evaluating global warming scenarios. I defend the intuition of neutrality, which answers a part of the question. It states that – other things being equal – it is ethically irrelevant whether or not additional people are added to a population. The argument against neutrality criticizes the intuition of neutrality as inconsistent. 

The contribution of this thesis is twofold: First, the framework of welfare economics, the intuition of neutrality, and the argument against neutrality are presented with formal rigour. Second, the formalizations will be used for a critical analysis of the argument against neutrality. Three ethical frameworks will be assumed -- the difference principle, average utilitarianism, and contractarianism --, and their relation to the explicit and hidden premises of the argument against neutrality will be investigated. 

The result will be that all three frameworks are compatible with the intuition of neutrality (or slightly modified versions); so the argument against neutrality does not hold within them. 

The analysis is built on several controversial philosophical views and does not necessarily disprove the argument against neutrality. Rather, it undermines the authority of the argument by pointing out the weakness of several premises within the three frameworks.  

I begin by briefly introducing the framework of welfare economics, which this thesis argues within. I then present in more detail the intuition of neutrality and the formal argument brought forward against it. The main part is dedicated to the development of three lines of argumentation in opposition to the argument against neutrality (thus defending the intuition of neutrality). I conclude with a systematic summary of the three lines of argumentation. 