\chapter{Critical analysis}
\section{Aggregation and justice}
\label{sec:obj1}

I will start by delivering some general criticism on Pareto domination and aggregation and then continue to examine their relation to justice-oriented welfare functions, specifically the Maximin and Leximin rule.   

When we say that a scenario Pareto dominates another scenario, we mean that at least one person is better off in this scenario than in the other while all other persons are at an equal level of well-being. The Pareto principle I have formulated as (P2) says that in such cases the first scenario has a higher welfare than the other one. This principle, as well as the extending requirement of Pareto efficiency (cf. \citeNP{osborne_1997}), find their due place in economics where the objective is the efficient allocation of scarce resources (\citeNP[p.~4]{samuelson_2010}; \citeNP{lange_2019}). However I doubt that they are suitable as ethical principles. Pareto efficiency has been criticized because the liberal paradox suggests that it may be incompatible with procedural elements of liberalism (see \citeNP{sen_1997}). But I believe that there is a more general problem with Pareto efficiency and even with Pareto domination: Consider a large population with one person whose well-being is much higher than the well-being of the others. Is it ethically desirable – is there a higher welfare – if the welll-being of this person is increased even more, while the well-being of the other persons remains the same? This can be intuitively doubted, and below I show some mildly convincing reasoning in favour of this doubt.  

A similar criticism applies to what I have called the fair aggregation principle. The fair aggregation principle is a combination of what can be called the simple aggregation principle – that general welfare is the simple sum of all individual well-being – with the additional requirement that distributions need be more equal to have a higher welfare. The principle is non-exhaustive: it does not tell us anything about populations with a higher sum of well-being and a lower equality, and it does not tell us anything about populations with a lower sum of well-being and a higher equality. But that is not a problem, since such populations do not play a relevant role in the counter-example to the intuition of neutrality.  

The problem with the requirement of equality is that, analytically, equality is a global criterion, which means that it somehow takes into account the well-being of every single person. This implies that a small decrease $\epsilon $ in well-being of the person who already is worst off can always be compensated by some large increase of equality within the rest of the population. This follows because otherwise the well-being of the worst-off person would completely determine the equality – which is intuitively plausible, but not incorporated in the conception of inequality measures. 

Now imagine three scenarios, all with the same people: In scenario X there is some utility distribution with lots of inequality. The person who is worst off in scenario X is called p. In scenario Y, the well-being of the worst-off person from scenario X is decreased by some very small amount $\epsilon$. Due to the globality of inequality, this can be compensated in terms of equality by improving the equality within all the other persons to a more or less drastical amount. Let us assume that such compensation has taken place, so that the overall equality in scenario Y is higher than in scenario X. With the usual inequality functions this will be possible without decreasing the sum of well-being (cf. \citeNP{ceriani_2012}). Let us further assume that in a third scenario Z all people are at the same level of well-being as the people in scenario Y, plus $\frac{\epsilon}{2}$. The general equality has not decreased in Z in comparison to Y. (Depending on the inequality function, it may even have increased, because the relative differences between the least well-off and the most well-off have decreased.) However the sum of well-being is increased in Z in comparison to X because the well-being of many persons has been increased by $\frac{\epsilon}{2}$ while the well-being of only one person has been decreased by $\frac{\epsilon}{2}$. As a consequence, both the sum of individual well-being and the equality are better in Z than in X, so according to the fair aggregation principle there is a higher welfare in Z than in X. At the same time, the worst-off person in X is even worse off in Z. This seems intuitively implausible and I will now present a theory which explains this implausibility. 

For this objection I will use as a specific welfare function the difference principle. The difference principle is a concept which is inferred from an analysis of justice. Its justification as the second principle of justice is given and extensively discussed in \cite[pp.~3-183]{rawls_2005}. Rivalling average utilitarianism, the difference principle is probably the most prominent and most widely accepted welfare function. In its core formula, the difference principle states that differences from socioeconomic equality are only permitted if they are to the benefit of the least advantaged \cite[p.~302]{rawls_2005}. This implies that society should aim to optimize the status of the least advantaged. The difference principle is therefore usually represented as a welfare function where general welfare is determined only by the well-being of the group with the lowest level of well-being. (Such representation commits a major error in ignoring the difference between primary goods and well-being as I discuss in \citeNP[p.~12f]{pomerenke_2017}. – But this does not bear upon the reasoning here, which is based solely on Pareto comparisons.) Whilst the difference principle refers to the least advantaged group – which makes sense in application – there is no mistake in referring to the least advantaged person for the sake of theory (cf. \cite[p.~98]{rawls_2005}). Because of its resemblance to the decision-theoretic rule of minimum maximization, this formulation of the principle has also been called the Maximin rule. (Although this labelling has been rightly criticized in \citeNP[p.~43.]{rawls_2001}.) 

\begin{Definition}{Difference principle / "Maximin"}{} 
$w(P) = \min_{p\in P} u(p)$
\end{Definition}

According to the difference principle in its Maximin version, both Pa\-re\-to domination and fair aggregation are false: Imagine that one person who is not the worst-off in either scenario is better off in the first scenario than in the second while all other persons are equally well off. Then Pareto domination requires that the first scenario has a higher welfare. The Maximin rule, however, states that both scenarios have the same welfare because the well-being of the worst-off person has not changed. And we have seen above that as a consequence of fair aggregation a scenario may be evaluated as having a higher welfare even if the worst-off person is even worse off – in strict contradiction to the difference principle. 

But the difference principle in its Maximin formulation has been designed as a simplification with the practical idea in mind that there will seldom or never be a comparison in which the least advantaged will have the exactly same level of well-being in both scenarios. Yet for the theoretical case of a such comparison a more elaborate rule than the Maximin rule has been developed (cf. \citeNP[p.~83]{rawls_2005}): It says that in the case that the least advantaged are at the same level in both scenarios, the second-least advantaged must be regarded. And if the second-least advantaged are also at the same level, then the third-least advantaged must be regarded, and so on. Because it resembles a lexicographical sorting algorithm, the extended rule is called the Leximin rule. It is most clearly formulated as a recursive selection function which outputs the better population of two populations whose members are sorted in ascending order according to their well-being:  

\begin{Definition}{Difference principle / "Leximin" selection function}{}
The set of the best population(s) of two populations $S = s_1, ..., s_n$ and $T = t_1, ..., t_n$ which are sorted in ascending by well-being, i. e.,
\begin{itemize}
\item $u(s_1) \leq ... \leq u(s_n)$,
\item $u(t_1) \leq ... \leq u(t_n)$,
\end{itemize}
is given by \\
lexiMin$(S=s_1...s_n, T=t_1...t_n) = $ \\
\[
  \begin{cases}
    \{S, T\} & \text{for } S = T = \varnothing \\
    \{S\} & \text{for } u(s_1) > u(t_1) \\
    \{T\} & \text{for } u(s_1) < u(t_1) \\
    \text{lexiMin}\left((s_2, ..., s_n), (t_2, ..., t_n)\right) & \text{for } u(s_1) = u(t_1)
  \end{cases}
\]
\end{Definition}

We can easily observe that – unlike the Maximin rule – the Leximin rule is compatible with Pareto domination: If all persons are equal in two scenarios except one who is better off in the second scenario, then the Leximin algorithm will recursively call another instance of the Leximin algorithm (where the worst-off from the outer instance will be disregarded), until an instance is called where the two persons in questions are the worst-off persons in their respective scenarios. This process automatically ensures Pareto domination. So at a second glance at the difference principle, it does not contradict but indeed rather support Pareto domination. This is in favour of the argument against neutrality.  

The same, however, cannot be said about the relation of the difference principle to the principle of fair aggregation. We have seen above that fair aggregation in some cases evaluates distributions as being better than a second distribution even though the worst-off person is better off in the second distribution. In such a case, the Leximin algorithm would stop in the first iteration, with a result equivalent to the result of the Maximin rule. The algorithm would not regard the improved well-being of all the other persons, because not only the Maximin rule but also the Leximin rule deem all general improvements irrelevant if they are to the disadvantage of the least advantaged. So for one major welfare function the "hard-to-doubt" premise of fair aggregation (P4) is false and the argument against neutrality cannot succeed. 

At the beginning of this section, two intuitive objections to the Pareto principle and the fair aggregation principle have been raised. The objection to the Pareto principle appeared to be supported by assuming the difference principle as a welfare function; however it turned out that the difference principle is only contradictory to the Pareto principle in its Maximin formulation, not in the more general and theoretically preferable formulation as the Leximin rule. The objection to the fair aggregation principle, however, was supported by both the Maximin and the Leximin formulation of the difference principle. As the fair aggregation principle is a necessary premise for the argument against neutrality, the argument therefore fails when the difference principle is assumed as a welfare function. 